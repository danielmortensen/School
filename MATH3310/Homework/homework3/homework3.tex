\documentclass{article}
\usepackage{caption}
\usepackage{subcaption}
\usepackage{amsmath}
\usepackage{amssymb}
\usepackage{mathtools}
\usepackage[margin=0.75in]{geometry}
\usepackage{fancyhdr}
\usepackage{xcolor}
\usepackage{tikz}
\usepackage[normalem]{ulem} % for strike through text
\setlength{\headheight}{0in}

\pagestyle{fancy}
\rhead{\today}
\lhead{Daniel Mortensen}
\chead{Homework 3}

\begin{document}
\noindent{\bf Problem 1:} Please prove that the sets $E = \{\mbox{even integers}\}$, $\mathbb{N}$, and $\mathbb{Z}$ are
infinite sets and that they have the same cardinality. 

\noindent\rule{\textwidth}{0.4pt}\vspace{0.05in}
\noindent{\it Claim 1: } The sets $E$, $\mathbb{N}$, and $\mathbb{Z}$ are infinite sets that have the same cardinality.
\\[0.05in] \noindent{\it Proof 1: } The claim is composed of several atomic statements: The first three claim that $E, \mathbb{N}$, and $\mathbb{Z}$ are infinite sets and the fourth claims that they have the same cardinality. These statements will be verified independently.
\\[0.1in] \noindent{\it Subclaim 1-1a: } The set $E$ is an infinite set.
\\[0.05in]\noindent{\it Subproof 1-1a: }  The set $E$ is an infinite set if there exists a bijection from a proper subset of $E$ to $E$. Let the set $B$ be a subset of $E$ such that $B = \{b \in E: 4\backslash b \}$. Additionally, let $f: E \rightarrow B$ such that $f(e) = 2\cdot e \ \forall e \in E$. In this proof, we will show that the function $f$ is a bijection between $E$ and $B$, and thus that $E$ is  infinite.
\\ \\
For a function to be bijective, it must be both surjective and injective. For the function $f$ to be surjective, $\forall b \in B, \exists e \in E \ni f(e) = b$. If $b \in B$, then we know that $b$ is divisible by 4 such that $\exists c \in \mathbb{Z} \ni 4\cdot c = b$ but 4 is also divisible by 2, implying that $2(2\cdot c) = b$. Note that $2\cdot c \in E$, which implies that $\exists e \in E \ \ni 2\cdot e = b$, and hence that $\exists e \in E \ \ni f(e) = b$, making $f$ surjective.
\\ \\
For a function to be injective, $f(a) = f(b)$ must imply that $a = b$.  Let $e_1, e_2 \in E$. This implies that $f(e_1) = f(e_2) \Rightarrow 2e_1 = 2e_2$ and consequently that $2e_1 - 2e_2 = 0$ and $2(e_1 - e_2) = 0$. But then $-e_2$ must be the additive inverse of $e_1$, making $e_1 = e_2$. Therefore, $f(a) = f(b) \Rightarrow a = b$, making $f$ injective. Because $f$ is both injective and surjective, it is also bijective.  Therefore, because $B$ is a proper subset of $E$ and $f$ a bijection between $B$ and $E$ the cardinality of set $E$ is infinite.
\\[0.1in] \noindent{\it Subclaim 1-1b: } The set $\mathbb{N}$ is an infinite set.
\\[0.05in]\noindent{\it Subproof 1-1b: }  The set $\mathbb{N}$ is an infinite set if there exists a bijection from a proper subset of $\mathbb{N}$ to $\mathbb{N}$. Let the set $B$ be a subset of $\mathbb{N}$ such that $B = \{b \in \mathbb{N}: 2\backslash b \}$. Additionally, let $f: \mathbb{N} \rightarrow B$ such that $f(n) = 2\cdot n \ \forall n \in \mathbb{N}$. In this proof, we will show that the function $f$ is a bijection between $\mathbb{N}$ and $B$, and thus that the cardinality of $\mathbb{N}$ is infinite.
\\ \\
For a function to be bijective, it must be both surjective and injective. For the function $f$ to be surjective, $\forall b \in B, \exists n \in \mathbb{N} \ni f(n) = b$. If $b \in B$, then we know that $b$ is divisible by 2 such that $\exists c \in \mathbb{Z} \ni 2\cdot c = b$. Note that $c \in \mathbb{N}$, implying that $\exists n \in \mathbb{N} \ \ni 2\cdot n = b$, and hence that $\exists n \in \mathbb{N} \ \ni f(n) = b$, making $f$ surjective.
\\ \\
For a function to be injective, $f(a) = f(b)$ must imply that $a = b$.  Let $n_1, n_2 \in \mathbb{N}$. This implies that $f(n_1) = f(n_2) \Rightarrow 2n_1 = 2n_2$ and consequently that $2n_1 - 2n_2 = 0$ and $2(n_1 - n_2) = 0$. But then $-n_2$ must be the additive inverse of $n_1$, making $n_1 = n_2$. Therefore, $f(a) = f(b) \Rightarrow a = b$, making $f$ injective. Because $f$ is both injective and surjective, it is also bijective.  Therefore, because $B$ is a proper subset of $\mathbb{N}$ and $f$ is a bijection between $B$ and $\mathbb{N}$, the cardinality of $\mathbb{N}$ is infinite.
\\[0.1in] \noindent{\it Subclaim 1-1c: } The set $\mathbb{Z}$ is an infinite set.
\\[0.05in]\noindent{\it Subproof 1-1c: } The set $\mathbb{Z}$ is an infinite set if there exists a bijection from a proper subset of $\mathbb{Z}$ to $\mathbb{Z}$. Let the set $B$ be a subset of $\mathbb{Z}$ such that $B = \{b \in \mathbb{Z}: 2\backslash b \}$. Additionally, let $f: \mathbb{Z} \rightarrow B$ such that $f(z) = 2\cdot z \ \forall z \in \mathbb{Z}$. In this proof, we will show that the function $f$ is a bijection between $\mathbb{Z}$ and $B$, and thus that $\mathbb{Z}$ is  infinite.
\\ \\
For a function to be bijective, it must be both surjective and injective. For the function $f$ to be surjective, $\forall b \in B, \exists z \in \mathbb{Z} \ni f(z) = b$. If $b \in B$, then we know that $b$ is divisible by 2 such that $\exists c \in \mathbb{Z} \ni 2\cdot c = b$, implying that $\exists n \in \mathbb{N} \ \ni f(n) = b$, making $f$ surjective.
\\ \\
For a function to be injective, $f(a) = f(b)$ must imply that $a = b$.  Let $z_1, z_2 \in \mathbb{Z}$. This implies that $f(z_1) = f(z_2) \Rightarrow 2z_1 = 2z_2$ and consequently that $2z_1 - 2z_2 = 0$ and $2(z_1 - z_2) = 0$. But then $-z_2$ must be the additive inverse of $z_1$, making $z_1 = z_2$. Therefore, $f(a) = f(b) \Rightarrow a = b$, making $f$ injective. Because $f$ is both injective and surjective, it is also bijective.  Therefore, because $B$ is a subset of $\mathbb{Z}$, and $f$ is a bijection between $B$ and $\mathbb{Z}$, the cardinality of $\mathbb{Z}$ is infinite.
\\[0.05in] \noindent{\it Lemma 1-2: } Let $A,B,C \in \{\text{Sets}\}$. Before we can prove that the sets $E$, $\mathbb{N}, \text{ and } \mathbb{Z}$ have the same cardinality, we must first prove that if the cardinality of set $A$ is equal to the cardinality of $B$, and the cardinality of $B$ is equal to the cardinality of $C$, then the cardinality of $A$ is equal to the cardinality of $C$.
\\[0.05in] Because the cardinality of $A$ is equal to the cardinality of $B$, there is a bijection between $A$ and $B$ such that $f:A\xrightarrow[\text{onto}]{1:1}B$. By the same logic, there is also a bijection between $B$ and $C$ such that $g:B\xrightarrow[\text{onto}]{1:1}C$. Define a separate function $h:A\rightarrow C$ where $h(a) = g(f(a))$. We will show that $h$ is a bijection from $A$ to $C$ and in doing so, that the cardinality of $A$ and the cardinality of $C$ are equal.
\\[0.05in]
First, we show that $h$ is surjective, or that $\forall a \in A, \  \exists c \in C \ni h(a) = c$. We know that $\forall b \in B, \ \exists a \in A \ni f(a) = b$ because $f$ is a bijection between $A$ and $B$.  We also know that $\forall c \in C, \ \exists b \in B \ni g(b) = c$. Therefore, $\forall c \in C,\ \exists a \in A \ni g(f(a)) = c$, implying that $\forall c \in C, \ \exists a \in A \ni h(a) = c$ and making $h$ surjective.
\\[0.05in]
Next we show that $h$ is injective, or that $h(a_1) = h(a_2) \Rightarrow a_1 = a_2 \forall a_1, a_2 \in A$ or, by its contrapositive, that $a_1 \ne a_2 \Rightarrow h(a_1) \ne h(a_2)$. Let $a_1 \ne a_2$, which implies that $f(a_1) \ne f(a_2)$ and finally that $g(f(a_1)) \ne g(f(a_2))$. The function $h$ is defined as $h(a) = g(f(a))$, and therefore, $a_1 \ni a_2 \Rightarrow h(a_1) \ne h(a_2)$, making $h$ injective. We have shown that $h$ is both injective and surjective and therefore is bijective.  The function $h$ is a bijection between $A$ and $C$ which implies that the cardinality of $A$ is the equal to the cardinality of $C$.
\\[0.1in] \noindent{\it Subclaim 1-2: } The sets $E$, $\mathbb{N}$, and $\mathbb{Z}$ have the same cardinality
\\[0.05in] \noindent{\it Subproof 1-2: } Two sets are said to have the same cardinality if there exists a bijection from one set to the other. We proved in Subproof 1-1c that there exists a bijection from $E$ to $\mathbb{Z}$, which implies that the cardinality of $E$ is equal to the cardinality of $\mathbb{Z}$.  We will now prove that there exists a bijection from $\mathbb{N}$ to $\mathbb{Z}$ and in doing so show that the cardinality of $\mathbb{Z}$ is equal to the cardinality of $\mathbb{N}$.
\\ \\
Let the function $f: \mathbb{Z} \rightarrow \mathbb{N}$ be a piecewise function where
\begin{equation*}
f(z) = 
\begin{cases}
2\cdot z & z >= 0 \\
-2\cdot z - 1 & z < 0 
\end{cases}.
\end{equation*}
The function $f$ maps the negative integers from $\mathbb{Z}$ to the odd integers in $\mathbb{N}$, and the non-negative integers from $\mathbb{Z}$ to the even integers in $\mathbb{N}$ although we will show that the function $f$ is bijective more rigorously by showing that $f$ is both surjective and injective.
\\ \\
For $f$ to be surjective, we just show that $\forall n \in \mathbb{N}, \ \exists z \in \mathbb{Z} \ni f(z) = n$. When $2\backslash n$, then $f(z) = 2z$ and becuse $2\backslash n$, then we know that there exists an element in $\mathbb{Z}$ such that $2z = n$. Furthermore, because the product of two positive values must be positive, then $z$ must be greater than zero, satisfying the piecewise constraints in $f$. If $n$ is odd, then $exists k \in \mathbb{N} \ \ni 2k - 1 = n$, implying that there also $ \exists k_{-} \in \mathbb{Z} \ \ni k = -k_{-}$. Therefore, $\exists k \in \mathbb{N} \ \ni 2k - 1 =n \Rightarrow \exists k_{-} \in \mathbb{Z} \backslash \mathbb{N} \ \ni -2k_{-} = n$ and hence, that $f(k_{-}) = n$. Therefore, $f$ is surjective.
\\ \\
To show that $f$ is injective, we show that $f(a) = f(b) \Rightarrow a = b$. If $2\backslash f(a)$, then $f(a) = 2\cdot a$ and $f(b) = 2\cdot b$ which implies that $2\cdot a - 2\cdot b = 0$ and consequently that $2(a - b) = 0$, which implies that $a = b$. If $2\nmid f(a)$, then $f(a) = -2\cdot a - 1$ and $f(b) = -2\cdot b - 1$ which implies that $-2\cdot a - 1 - (-2\cdot b - 1) = 0$ and consequently that $-2\cdot a - 1 + 2\cdot b + 1 = 0$ which is equivalent to $-2(a - b) = 0$, and implies that $a = b$ which in turn implies that $f$ is injective.
\\ \\
We have shown that the cardinality of $\mathbb{Z}$ is equal to the cardinality of $E$ and that the cardinality of $\mathbb{Z}$ is also equal to the cardinality of $\mathbb{N}$. By lemma 1-2, we also know that the cardinality of $E$ is equal to the cardinality of $\mathbb{N}$, implying that $\mathbb{Z}$, $\mathbb{N}$, and $E$ all have the same cardinality.
\\[0.05in] \noindent\rule{\textwidth}{0.4pt}\vspace{0.05in}
\\[0.1in] \noindent{\bf Problem 2:} Please prove that the set $(0,1) = \{x \in \mathbb{R}: 0 < x < 1\}$ is infinite, but does not have the same cardinality as $\mathbb{N}$ (and hence not the same as $E$ and $\mathbb{Z}$)
\\[0.05in]\noindent\rule{\textwidth}{0.4pt}\vspace{0.05in}
\noindent{\it Claim 2: } The set $(0,1) = \{ \mathbb{R} : 0 < x < 1 \}$ is infinite but does not have the same cardinality as $\mathbb{N}$
\\[0.05in]\noindent{\it Proof 2: } The claim is composed of two atomic statements: The first claims that the set $(0,1) = \{x \in \mathbb{R}: 0 < x < 1\}$ is infinite and the second claims that the same set does not have the same cardinality as $\mathbb{N}$. The two statements will be verified independently.
\\[0.1in] \noindent{\it Subclaim 2-1: } The set $(0,1) = \{x \in \mathbb{R}: 0 < x < 1 \}$ is infinite 
\\[0.05in]\noindent{\it Subproof 2-1: } To show that the set $(0,1)$ is infinite, we show that the function $f:(0,1) \rightarrow (0,\frac{1}{2})$ is bijective where $f$ is defined as $f(a) = \frac{a}{2}$. To show that $f$ is bijective, we first show that $f$ is surjective, that is $\forall b \in (0,\frac{1}{2}), \exists a \in (0,1) \ni f(a) = b$. From the definition of $f$, we know that $f(a) = b \Rightarrow \frac{a}{2} = b$, which also implies that $2b = a$ and thus (dumbass reader) there does exist an $a$ such that $f(a) = b$ and $f$ is surjective.  
\\ \\
We next show that $f$ is injective, that is $f(a) = f(b) \Rightarrow a = b \ \forall a \in (0,1), \ b \in (0,\frac{1}{2})$. From the definition of $f$, $f(a) = f(b) \Rightarrow \frac{a}{2} = \frac{b}{2}$, which implies that $\frac{a}{2}\cdot 2 = \frac{b}{2}\cdot 2$ and finally that $a = b$. Therefore, if $f(a) = f(b)$, then $a = b$ and $f$ is injective. 
\\ \\
Because $f$ is both surjective and injective, $f$ is also bijective, implying that there exists a bijection between $(0,\frac{1}{2}$ and $(0,1)$.  Furthermore, because $(0,\frac{1}{2})$ is a proper subset of $(0,1)$, the set $(0,1)$ is infinite.
\\[0.1in] \noindent{\it Subclaim 2-2: } The set $(0,1) = \{x \in \mathbb{R}: 0 < x < 1 \}$ does not have the same cardinality as $\mathbb{N}$.
\\[0.05in] \noindent{\it Subproof 2-2: } To show that the set $(0,1)$ does not have the same cardinality as $\mathbb{N}$, we must show that there cannot exist a bijection between the two sets. In doing so, it is sufficient to show that there cannot exist a surjection between the two sets. Suppose there exists a surjection $f: \mathbb{N} \rightarrow (0,1)$ where the values in the domain and codomain are organized into a table such that 
\begin{center}\begin{tabular}{c|c}
	$x$ (Domain) & $f(x)$ (codomain) \\ \hline
	1 & $ f(1)$ \\
	2 & $ f(2)$ \\
	3 & $ f(3)$ \\
	4 & $ f(4)$ \\
	$\vdots$ & $\vdots$ \\
\end{tabular}.\end{center}
Because each element in the codomain of $f$ is in the set $(0,1)$, each value can be expressed in decimal form, where each decimal place contains a value from zero to nine such that $f(i) = 0.\hspace{0.02in} x_{i,1} \hspace{0.02in} x_{i,2} \hspace{0.02in} x_{i,3}\hspace{0.02in}x_{i,4}\cdots x_{i,n}$, where $n$ represents the number of values in the decimal representation and can be any arbitrary value in the set of all positive integers. Let $x_{i,j}$ be the value of the $j^{\text{th}}$ decimal place for $f(i) \ \forall i \in \mathbb{N}$. The table representing the domain and codomain of $f$ can be rewritten where
\begin{center}\begin{tabular}{c|c}
	$x$ (Domain) & $f(x)$ (codomain) \\ \hline
	1 & $ 0.\hspace{0.02in} x_{1,1} \hspace{0.02in} x_{1,2} \hspace{0.02in} x_{1,3}\hspace{0.02in}x_{1,4}\cdots x_{1,n_1}$ \\
	2 & $ 0.\hspace{0.02in} x_{2,1} \hspace{0.02in} x_{2,2} \hspace{0.02in} x_{2,3}\hspace{0.02in}x_{2,4}\cdots x_{2,n_2}$ \\
	3 & $ 0.\hspace{0.02in} x_{3,1} \hspace{0.02in} x_{3,2} \hspace{0.02in} x_{3,3}\hspace{0.02in}x_{3,4}\cdots x_{3,n_3}$ \\
	4 & $ 0.\hspace{0.02in} x_{4,1} \hspace{0.02in} x_{4,2} \hspace{0.02in} x_{4,3}\hspace{0.02in}x_{4,4}\cdots x_{4,n_4}$ \\
	$\vdots$ & $\vdots$ \\
\end{tabular}.\end{center}
Let $\Phi(i,j)$ map $(i,j)$ to one if $x_{i,j} = 0$, and zero otherwise such that 
\begin{equation*}
	\Phi(i,j) = \begin{cases} 1 & x_{i,j} = 0 \\ 0 & x_{i,j} \ne 1 \end{cases}.
\end{equation*}
Let $a\in (0,1)$ which is constructed using the diagonal elements from the codomain of $f$ as shown in Table \ref{tab:question2:diag} so that $a = 0.\Phi(1,1)\Phi(2,2)\Phi(3,3)\cdots\Phi(n,n)$. 
\begin{table}[h!]\begin{center}
\begin{tabular}{c|c}
	$x$ (Domain) & $f(x)$ (codomain) \\ \hline
	1 & $ 0.\hspace{0.02in} \textcolor{red}{x_{1,1}} \hspace{0.02in} x_{1,2} \hspace{0.02in} x_{1,3}\hspace{0.02in}x_{1,4}\cdots x_{1,n_1}$ \\
	2 & $ 0.\hspace{0.02in} x_{2,1} \hspace{0.02in} \textcolor{red}{x_{2,2}} \hspace{0.02in} x_{2,3}\hspace{0.02in}x_{2,4}\cdots x_{2,n_2}$ \\
	3 & $ 0.\hspace{0.02in} x_{3,1} \hspace{0.02in} x_{3,2} \hspace{0.02in} \textcolor{red}{x_{3,3}}\hspace{0.02in}x_{3,4}\cdots x_{3,n_3}$ \\
	4 & $ 0.\hspace{0.02in} x_{4,1} \hspace{0.02in} x_{4,2} \hspace{0.02in} x_{4,3}\hspace{0.02in}\textcolor{red}{x_{4,4}}\cdots x_{4,n_4}$ \\
	$\vdots$ & $\vdots$ \\
\end{tabular}\end{center}
\caption{Diagonal elements}
\label{tab:question2:diag}
\end{table}
The element $a$ cannot be equal to $f(1)$. If the first entry of $f(1)$ is equal to $0$, then the first value of $a$ is equal to 1.  If the first value of $f(1)$ is not equal to 0, then the first decimal entry of $a$ is equal zero, so that $f(1) \neq a$. The same logic applies for any $f(i), \ i \in \mathbb{N}$, implying that $a$ has no preimage despite being an element in $(0,1)$.  Therefore, $f$ is not a surjection, which contradicts the original assumption that $f$ was surjective between $\mathbb{N}$ and $(0,1)$. Therefore, there cannot exist any surjection (and by extension, any bijection) between $\mathbb{N}$ and $(0,1)$ which also means that the cardinality of $\mathbb{N}$ is not equal to the cardinality of $(0,1)$. 
\\[0.05in] \noindent\rule{\textwidth}{0.4pt}\vspace{0.05in}

\noindent{\bf Problem 3: } Recall $\implies, \vee, \wedge,$ and $\nabla$ are binary logical operations on $\mathcal{M}$ we've studied.  Recall $\neg$ is a (the only) logical \emph{operator} (or \emph{unary operation}) on $\mathcal{M}$ we've studied.  Please determine, with justification, the number of different binary logical operations on $\mathcal{M}$ there can be.  Also, determine the number of logical operators there can be.
\\\rule{\textwidth}{0.4pt} 
\\[0.05in]\noindent{\it Claim 3-1:} There are 16 possible binary operations on $\mathcal{M}$
\\[0.05in]\noindent{\it Proof 3-1:} A binary operation is essentially a mapping $f:(\{\text{True}, \text{False}\} \times \{\text{True}, \text{False}\}) \rightarrow \{\text{True}, \text{False}\}$, where the domain of $f$ has four elements: $(\text{True}, \text{True}), (\text{True}, \text{False}),(\text{False}, \text{True})$ and $(\text{False}, \text{False})$. The image of each element in the domain can be either True, or False such that
\begin{equation*}
\centering
\begin{tabular}{c | c | c}
P & Q & f(P,Q) \\ \hline
True & True & \{True, False \} \\
True & False & \{True, False \} \\
False & True & \{True, False \} \\
False & False & \{True, False \}
\end{tabular}.
\end{equation*}
Let the set $\mathcal{S}$ include all possible combinations of the set $\{(P, Q) \times f(P,Q)\}$. Note that for each row in the table, the number of elements in $\mathcal{S}$ doubles, implying that there are $2^4 = 16$ modes of behavior for a binary operation in $\mathcal{M}$. Each mode of behavior can be represented by at most one binary operation and so, there can be at most $16$ binary operations on $\mathcal{M}$.
\\[0.05in]\noindent{\it Claim 3-2:} There can exist up to four binary operators on $\mathcal{M}$
\\[0.05in]\noindent{\it Proof 3-2:} A binary operator on $\mathcal{M}$ is defined as a function $f:\mathcal{M} \rightarrow \mathcal{M}$. A truth table for such a function could be written as 
\begin{equation*}
\centering
\begin{tabular}{c | c}
P & f(P) \\ \hline
True & \{True, False\} \\
False & \{True, False\} 
\end{tabular}.
\end{equation*}
Note that there are two rows in the truth table for a binary operator on $\mathcal{M}$. By the same logic as Proof 3-1, this implies that the number of combinations of $(P, f(P))$ is equal to $2^2 = 4$ and because a binary operator on $\mathcal{M}$ can only represent one combination, there are four possible binary operators on $\mathcal{M}$.
\\[0.05in]\rule{\textwidth}{0.4pt} 
\\[0.05in]\noindent{\bf Problem 4: } Turn the logical system $(\mathcal{M}, \Phi, \wedge, \vee, \implies, \Longleftarrow, \Longleftrightarrow, \neg)$ into a purely algebraic system over $(\mathbb{Z}_2, +,\cdot)$ with, for $x \in \mathcal{M}$, $\Phi(x) = 0$ if $x$ is true and $\Phi(x) = 1$ if $x$ is false.  Use the logical system to show that any statement of the form $[\neg P \wedge ((\ neg P \implies Q) \wedge \neg Q)] \implies P$ is a tautology.
\\[0.05in]\rule{\textwidth}{0.4pt} 
The logical system $(\mathcal{M}, \Phi, \wedge, \vee, \implies, \Longleftarrow, \Longleftrightarrow, \neg)$ can be expressed as an algebraic system over $(\mathbb{Z}_2, + ,\cdot)$ by expressing $\Phi, \wedge, \vee, \implies, \Longleftarrow, \Longleftrightarrow,$ and $\neg$ in term of $+$ and $\cdot$ in $\mathbb{Z}_2$.
\\[0.05in]\noindent{\it Claim 4-1: } The operator $\neg$ can be expressed in terms of $+$ and $\cdot$ from $\mathbb{Z}_2$. 
\\[0.05in]\noindent{\it Proof 4-1: } Recall how the multiplication table for $\mathbb{Z}_2$ is
\begin{equation*}
\begin{tabular}{c | c | c}
+ & 0 & 1 \\ \hline
0 & 0 & 1 \\ 
1 & 1 & 0 \\
\end{tabular}.
\end{equation*}
and that the column corresponding to $1$ can be written in a truth table format as
\begin{equation*}
\begin{tabular}{c | c}
$P$ & $P+1$ \\ \hline
0 & 1 \\ 
1 & 0 \\
\end{tabular}.
\end{equation*}
where $0$ denotes `True' and $1$ denotes `False'.  Note how the truth table for $P$ and $P+1$ is the same as the truth table for $\neg$, implying that $\neg$ can be expressed algebraically such that $\neg P \equiv P + 1$ in $\mathbb{Z}_2$.
\\[0.05in]\noindent{\it Claim 4-2: } The operation $\vee$ can be expressed in terms of $+$ and $\cdot$ from $\mathbb{Z}_2$. 
\\[0.05in]\noindent{\it Proof 4-2: } The truth table for the operation $\vee$ is 
\begin{equation*}
\begin{tabular}{c | c | c | c | c | c}
$P$          & $\Phi(P)$ & $Q$          & $\Phi(Q)$ & $P\vee Q$ & $\Phi(P\vee Q)$\\ \hline
\text{True}  &    0    & \text{True}  &   0      & True        &      0           \\ 
\text{False} &    1    & \text{True}  &   0      & True        &      0           \\
\text{True}  &    0    & \text{False} &   1      & True        &      0           \\
\text{False} &    1    & \text{False} &   1      & False       &      1
\end{tabular}.
\end{equation*}
Note how the table for multiplication in $\mathbb{Z}_2$ is
\begin{equation*}
\begin{tabular}{c | c | c}
+ & 0 & 1 \\ \hline
0 & 0 & 0 \\ 
1 & 0 & 1 \\
\end{tabular}.
\end{equation*}
which can be expressed in terms of $\Phi(\cdot)$ and reorganized as a truth table such that
\begin{equation*}
\begin{tabular}{c | c | c}
$\Phi(P)$ & $\Phi(Q)$ & $\Phi(P)\cdot \Phi(Q)$ \\ \hline
 0        &    0      & 0 \\
 1        &    0      & 0 \\
 0        &    1      & 0 \\
 1        &    1      & 1 \\
\end{tabular},
\end{equation*}
which is equivalent to the truth table for $\vee$, therefore $\vee$ can be expressed algebraically in terms of $(\mathbb{Z}_2, + , \cdot)$.
\\[0.05in]\noindent{\it Claim 4-3: } The operation $\wedge$ can be expressed in terms of $+$ and $\cdot$ from $\mathbb{Z}_2$. 
\\[0.05in]\noindent{\it Proof 4-3: } De Morgan's Theorem states that $\neg(P \wedge Q) \Longleftrightarrow \neg P \vee \neg Q$, implying that $P \wedge Q \Longleftrightarrow \neg(\neg P \vee \neg Q)$. Note that the expression $P \wedge Q$ is equivalent to another expression which contains only the $\neg$ operator and the $\vee$ operation which we have shown can be expressed algebraically in $(\mathbb{Z}_2, +, \cdot)$ which implies that $\wedge$ can also be expressed as such.
\\[0.05in]\noindent{\it Claim 4-4: } The operation $\implies$ can be expressed in terms of $+$ and $\cdot$ from $\mathbb{Z}_2$. 
\\[0.05in]\noindent{\it Proof 4-4: } From homework 2, recall that $P \implies Q$ can be expressed in terms of the logical statement $P \vee \neg Q$. Because the operator $\vee$ and the operation $\neg$ can be expressed algebraically, then $P \vee \neg Q$ can also be expressed as such, implying that $P \implies Q$ can be written in terms of the algebraic system $(\mathbb{Z}_2, \cdot, +)$.
\\[0.05in]\noindent{\it Claim 4-5: } The operation $\Longleftarrow$ can be expressed in terms of $+$ and $\cdot$ from $\mathbb{Z}_2$. 
\\[0.05in]\noindent{\it Proof 4-5: } The truth table for the operator $P \Longleftarrow Q$ is given as
\begin{equation*}
\begin{tabular}{c|c|c}
$P$ & $Q$ & $P \Longleftarrow Q$ \\ \hline
True & True & True   \\
True & False & False \\
False & True & True  \\
False & False & True \\
\end{tabular}
\end{equation*}
and can be expressed as a logical statement $Q\vee \neg P$ as shown by their truth tables
\begin{equation*}
\begin{tabular}{c|c|c|c}
$P$ & $Q$ & $P \Longleftarrow Q$ & $Q\vee \neg P$ \\ \hline
True & True   & True  & True  \\
True & False  & False & False \\
False & True  & True  & True  \\
False & False & True  & True  \\
\end{tabular}.
\end{equation*}
The logical operator $\vee$ and the operation $\neg$ have already been shown to be expressable in the algebraic system $(\mathbb{Z}_2,+,\cdot)$ which implies that $\Longleftarrow$ can as well because $\Longleftarrow$ can be expressed in terms of $\vee$ and $\neg$. 
\\[0.05in]\noindent{\it Claim 4-6: } The operation $\Longleftrightarrow$ can be expressed in terms of $+$ and $\cdot$ from $\mathbb{Z}_2$. 
\\[0.05in]\noindent{\it Proof 4-6: } The truth table for $\Longleftrightarrow$ is given as
\begin{equation*}
\begin{tabular}{c | c | c}
$P$ & $Q$ & $P \Longleftrightarrow Q$ \\ \hline
True & True & True \\
False & True & False \\
True & False & False \\
False & False & True \\
\end{tabular}
\end{equation*}
which implies that
\begin{equation*}
\begin{tabular}{c | c | c}
$\Phi(P)$ & $\Phi(Q)$ & $\Phi(P \Longleftrightarrow Q)$ \\ \hline
0 & 0 & 0 \\
1 & 0 & 1 \\
0 & 1 & 1 \\
1 & 1 & 0 \\
\end{tabular}
\end{equation*}
and can be expressed logically as $(P\wedge Q)\vee (\neg P \wedge \neg Q)$. The logical expression for $\Longleftrightarrow$ is built upon the operators/operations $\wedge, \vee$ and $\neg$ which can be expressed algebraically in $(\mathbb{Z}_2, + , \cdot)$.  Therefore, $\Longleftrightarrow$ can also be expressed algebraically in $(\mathbb{Z}_2, +, \cdot)$.
\\[0.05in]\noindent{\it Claim 4-7: } The statement $[\neg P \wedge ((\neg P \Longrightarrow Q) \wedge \neg Q)] \Longrightarrow P$ is a tautology.
\\[0.05in]\noindent{\it Proof 4-7: } Begin by defining the following expressions
\begin{align}
\neg P &\equiv P + 1 \label{eqn:1}\\
P \implies Q &\equiv Q(P + 1) \label{eqn:2}\\
P \wedge Q &\equiv (P + 1)(Q + 1) + 1 \label{eqn:4} \\
1 + 1 &\equiv 0 \text{ in } \mathbb{Z}_2  \label{eqn:5}\\
PP &\equiv P \label{eqn:6}
\end{align}.
Next, begin replacing all $\neg$ operations such that
\begin{equation*}
[\neg P \wedge ((\neg P \Longrightarrow Q) \wedge \neg Q)] \Longrightarrow P \equiv [(P + 1)\wedge (((P + 1) \implies Q)\wedge (Q+1))]\implies P
\end{equation*}
Next, we replace the first use of the implies operation with it's mathematical definition from \eqref{eqn:2}, yielding 
\begin{equation*}
[\neg P \wedge ((\neg P \Longrightarrow Q) \wedge \neg Q)] \Longrightarrow P \equiv[(P + 1)\wedge (((P + (1 + 1))Q)\wedge (Q+1))]\implies P
\end{equation*}
But because $1 + 1 = 0$ in $\mathbb{Z}_2$, we can also say that
\begin{equation*}
[\neg P \wedge ((\neg P \Longrightarrow Q) \wedge \neg Q)] \Longrightarrow P \equiv[(P + 1)\wedge ((PQ) \wedge (Q+1))]\implies P.
\end{equation*}
The next step is to substitute the algebraic definition for $\wedge$ for the second $\wedge$ operation such that
\begin{equation*}
[\neg P \wedge ((\neg P \Longrightarrow Q) \wedge \neg Q)] \Longrightarrow P \equiv[(P + 1)\wedge ((PQ + 1)(Q+1+1))]\implies P.
\end{equation*}
which reduces to
\begin{equation*}
[\neg P \wedge ((\neg P \Longrightarrow Q) \wedge \neg Q)] \Longrightarrow P \equiv[(P + 1)\wedge (PQQ + Q)]\implies P.
\end{equation*}
Note, that $QQ \equiv Q$ because when $Q = 1, QQ = 1$ and when $Q = 0$, $QQ = 0$, therefore the previous expression can be simplified again as
\begin{equation*}
[\neg P \wedge ((\neg P \Longrightarrow Q) \wedge \neg Q)] \Longrightarrow P \equiv[(P + 1)\wedge (PQ + Q)]\implies P.
\end{equation*}
Next, we substitute the last remaining $\wedge$ operation for its algebraic definition to obtain
\begin{equation*}
[\neg P \wedge ((\neg P \Longrightarrow Q) \wedge \neg Q)] \Longrightarrow P \equiv[(P + 1 + 1)(PQ + Q + 1)]\implies P.
\end{equation*}
which simplifies to
\begin{equation*}
[\neg P \wedge ((\neg P \Longrightarrow Q) \wedge \neg Q)] \Longrightarrow P \equiv[(PPQ + PQ + P)]\implies P
\end{equation*}
and alternatively as 
\begin{equation*}
[\neg P \wedge ((\neg P \Longrightarrow Q) \wedge \neg Q)] \Longrightarrow P \equiv[(PQ + PQ + P)]\implies P.
\end{equation*}
Note that when PQ = 1, PQ + PQ = 0, and when PQ = 0, PQ + PQ = 0, therefore, PQ + PQ = 0 can be applied to the previous expression as
\begin{equation*}
[\neg P \wedge ((\neg P \Longrightarrow Q) \wedge \neg Q)] \Longrightarrow P \equiv P\implies P
\end{equation*}
and finally, the algebraic definition of $\implies$ is used to show that
\begin{equation*}
[\neg P \wedge ((\neg P \Longrightarrow Q) \wedge \neg Q)] \Longrightarrow P \equiv P(P + 1).
\end{equation*}
Because the expression P(P+1) = 0 when P = 1, and P(P+1) = 0 when P = 0, we can say that
\begin{equation*}
[\neg P \wedge ((\neg P \Longrightarrow Q) \wedge \neg Q)] \Longrightarrow P \equiv 0.
\end{equation*}
which implies that the statement $[\neg P \wedge ((\neg P \Longrightarrow Q) \wedge \neg Q)] \Longrightarrow P$ is a tautology as the statement is always true.
\\[0.05in]\rule{\textwidth}{0.4pt}

\end{document}
