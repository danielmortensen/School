\documentclass{article}
\usepackage{caption}
\usepackage{subcaption}
\usepackage{amsmath}
\usepackage{amssymb}
\usepackage{mathtools}
\usepackage[margin=0.75in]{geometry}
\usepackage{fancyhdr}
\usepackage{xcolor}
\usepackage{tikz}
\usetikzlibrary{backgrounds}
\usetikzlibrary{calc}
\usepackage[normalem]{ulem} % for strike through text
\setlength{\headheight}{0in}

\newcommand{\problemsep}{\leavevmode\\[0.05in] \rule[\baselineskip/4]{\textwidth}{1pt} \\[0.005in] \rule[\baselineskip]{\textwidth}{1pt}\vspace{-\baselineskip}\leavevmode\\[0.05in]}
\newcommand{\statementsep}{\leavevmode\\[0.005in] \rule[\baselineskip/4]{\textwidth}{0.4pt}\leavevmode\\[0.005in]}
\pagestyle{fancy}
\rhead{\today}
\lhead{Daniel Mortensen}
\chead{Homework 4}

\begin{document}
\noindent\underline{Problem 1}: Consider, again, the Petersen graph; thatis, the graph from HW5.2: $G = (V,E), V=\{2-\text{sets of } [5]\}$, vertices adjacent if and only if they are disjoint. Define $\tau(H)$ to be the number of spanning trees of graph $H$. 
\begin{enumerate}
	\item Compute $\tau(G)$
	\item Let $e$ be any edge of $G$, and define $G'$ to be $g - e$  ($G$ with the edge $e$ deleted); find all possible values of $\tau(G')$.
	\item Let $G''$ be $G$ with two edges deleted. Determine all possible values of $\tau(G'')$.
	\item Recall that we proved this in class: {\it The maximum distance between anypair of vertices of $G$ is 2}. Please prove this using Theorem A.
\end{enumerate}
\statementsep
\begin{enumerate}
	\item We compute $\tau(G)$ as the determinant of the laplacian of the Petersen graph with one column/row removed. The laplacian of the Petersen graph is:
		\begin{equation*}
			\begin{tabular}{c | c | c | c | c |c | c | c | c | c | c |}
				   & 1,2 & 1,3 & 1,4 & 1,5 & 2,3 & 2,4 & 2,5 & 3,4 & 3,5 & 4,5 \\ \hline 
				1,2& 3   & 0   & 0   & 0   & 0   & 0   & 0   & -1   & -1   & -1   \\ \hline
				1,3& 0   & 3   & 0   & 0   & 0   & -1   & -1   & 0   & 0   & -1   \\ \hline
				1,4& 0   & 0   & 3   & 0   & -1   & 0   & -1   & 0   & -1   & 0   \\ \hline
				1,5& 0   & 0   & 0   & 3   & -1   & -1   & 0   & -1   & 0   & 0   \\ \hline
				2,3& 0   & 0   & -1   & -1   & 3   & 0   & 0   & 0   & 0   & -1   \\ \hline
				2,4& 0   & -1   & 0   & -1   & 0   & 3   & 0   & 0   & -1   & 0   \\ \hline
				2,5& 0   & -1   & -1   & 0   & 0   & 0   & 3   & -1   & 0   & 0   \\ \hline
				3,4& -1   & 0   & 0   & -1   & 0   & 0   & -1   & 3   & 0   & 0   \\ \hline
				3,5& -1   & 0   & -1   & 0   & 0   & -1   & 0   & 0   & 3   & 0   \\ \hline
				4,5& -1   & -1   & 0   & 0   & -1   & 0   & 0   & 0   & 0   & 3   \\ \hline
			\end{tabular}
		\end{equation*}
		as the laplacian is defined as $D - A$ where $D$ is the degree matrix (i.e. a diagonal matrix where each diagonal element is the degree of the corresponding vertex) and $A$ is the adjacency matrix.  We know from homework 6 that the degree of each vertex in the Petersen graph is $3$, which yields the matrix given above. We also know that the number of spanning trees is given as the determinant of the laplacian with row and column $i$ removed.  This yields 2,000 spanning trees.
	\item In this section we compute $\tau(G')$. First, note that the number of spanning trees will be the same regardless of which edge is removed. Therefore, we only need compute $\tau(G')$ for one instance of $G'$. By removing one edge and computing $\tau(G')$, the number of possible trees for $G'$ is 800.
	\item In this section we compute $\tau(G'')$.  First, note that we only need compute the number of trees for two cases.  The first case occures when two edges that connect to the same vertex are removed, and the second happens when the two edges connect to different vertices. Computing $\tau(G'')$ for the first yields $\tau(G'') = 240$, and $\tau(G'')$ for the second case is equal to $300$.
	\item In this section, we show that the maximum number of steps between any two vertices is two.  From Theorem $A$, we know that {\it If $A$ is the adjacency matrix of $G$, entry (i,j) in $A^k$ is the number of walks of length $k$ from $v_i$ to $v_j$.} We compute $A + A^2$ of adjacency matrix for the Petersen graph which yields a matrix of all ones, except on the diagonal where the values are three.  This implies that the distance between two vertices is either one or two so that the maximum distance between any two vertices is two.
\end{enumerate}
\problemsep
\noindent\underline{Problem 2}: 
Use Theorem $A$ and the trace of the appropriate power of the adjacency matrix of the graph below to determine how many "triangles" the graph given in the problem statement has.
\statementsep
The adjacency matrix of the graph given in the problem statement is:
\begin{equation*}
	\begin{tabular}{c | c | c | c | c | c | c | c | c | c | c | c |}
		        & $v_1$ & $v_2$ & $v_3$ & $v_4$ & $v_5$ & $v_6$ & $v_7$ & $v_8$ & $v_9$ & $v_{10}$ & $v_{11}$ \\ \hline
		$v_1$   & 0     & 1     & 0     & 0     & 1     & 0     & 1     & 1     & 1     & 1        & 0        \\ \hline
		$v_2$   & 1     & 0     & 1     & 0     & 0     & 1     & 0     & 1     & 0     & 0        & 0        \\ \hline
		$v_3$   & 0     & 1     & 0     & 1     & 0     & 0     & 1     & 0     & 1     & 0        & 0        \\ \hline
		$v_4$   & 0     & 0     & 1     & 0     & 1     & 0     & 0     & 1     & 0     & 1        & 0        \\ \hline
		$v_5$   & 1     & 0     & 0     & 1     & 0     & 1     & 0     & 0     & 1     & 0        & 0        \\ \hline
		$v_6$   & 0     & 1     & 0     & 0     & 1     & 0     & 0     & 0     & 0     & 0        & 1        \\ \hline
		$v_7$   & 1     & 0     & 1     & 0     & 0     & 0     & 0     & 0     & 0     & 0        & 1        \\ \hline
		$v_8$   & 1     & 1     & 0     & 1     & 0     & 0     & 0     & 0     & 0     & 0        & 1        \\ \hline
		$v_9$   & 1     & 0     & 1     & 0     & 1     & 0     & 0     & 0     & 0     & 0        & 1        \\ \hline
		$v_{10}$& 1     & 0     & 0     & 1     & 0     & 0     & 0     & 0     & 0     & 0        & 1        \\ \hline
		$v_{11}$& 0     & 0     & 0     & 0     & 0     & 1     & 1     & 1     & 1     & 1        & 0        \\ \hline
	\end{tabular}
\end{equation*}
We also know that a "triangle" is a cycle with three steps, which is the number of of walks of length $3$ from one vertex back to itself, which is divided by 6 because each triangle is counted once for both the forward and backward direction, and for each starting edge of the path (which is three in this case).  Therefore, the number of triangles in the graph is 
\begin{equation*}
	\text{Tr}(A^3)/6 = 2
\end{equation*}
where Tr$(\cdot)$ is the trace of the resulting matrix.  We use the trace because we desire to add all cycles of length three (or paths that start and end on the same vertex) which is located along the diagonal of $A^3$.
\end{document}
